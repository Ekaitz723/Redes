%%%%%%%%%%%%%%%%%%%%%%%%%%%%%%%%%%%%%%%%%
% Cleese Assignment
% Structure Specification File
% Version 1.0 (27/5/2018)
%
% This template originates from:
% http://www.LaTeXTemplates.com
%
% Author:
% Vel (vel@LaTeXTemplates.com)
%
% Modified by:
% Ekaitz Arriola Garcia
%
% License:
% CC BY-NC-SA 3.0 (http://creativecommons.org/licenses/by-nc-sa/3.0/)
% 
%%%%%%%%%%%%%%%%%%%%%%%%%%%%%%%%%%%%%%%%%

%----------------------------------------------------------------------------------------
%	Sortak eta bestelako dokumentu-konfigurazioak
%----------------------------------------------------------------------------------------

\usepackage{lastpage}           % Plama-oinaren azken orrialdearen zenbakia xedatzeko

\usepackage{graphicx}           % Irudientzat beharrezkoa

\setlength\parindent{0pt}       % Paragrafoetatik koska guztiak kentzen ditu

\usepackage[most]{tcolorbox}    % Orrialde artean zatitzen den kutxentzat beharrezkoa

\usepackage{booktabs}           % Taulen lerro horizontal hoberentzat beharrezkoa

\usepackage{listings}           % Kodearentzat beharrezkoa

\usepackage{etoolbox}           % If adierazpenentzat beharrezkoa

\usepackage{lipsum}             % Testu betegarriarentzat beharrezkoa

\usepackage{xparse}             % Parametroekin gauza gehiago egiteko aukera ematen du

\usepackage{units}              % Zatidura politagoak (\nicefrac{3}{8})

\usepackage{xypic}              % Grafikoak fletxeekin

\usepackage{contour}            % Hizkientzako koloredun kontornoa

\usepackage{listings}           % Zerrendatuak
\usepackage[listings]{tcolorbox}
\tcbuselibrary{listings}
\tcbset{listing engine=listings}
\tcbuselibrary{listingsutf8}

\usepackage{color}
\definecolor{gray97}{gray}{.97}
\definecolor{gray75}{gray}{.75}
\definecolor{gray45}{gray}{.45}

\usepackage[
	section,
	cache=true,
	cachedir={\detokenize{~/.cache/minted}},
	newfloat=false
]{minted} % Kode hobeagoa

\usepackage{caption, subcaption} % Testu eta azpi-testu gehiagorako aukera ematen du

\usepackage[x11names]{xcolor} % Kolore-izen batzuk

% Azpi-fitxategiak
\usepackage{subfiles} % Best loaded last in the preamble
\usepackage{import}

%----------------------------------------------------------------------------------------
%	Marjinak
%----------------------------------------------------------------------------------------

\usepackage{geometry} % Plama dimentsio eta marjinak doitzeko beharrezkoa

\geometry{
	paper=a4paper, % Aldatu plama mota
	top=3cm, % Goiko marjina
	bottom=3cm, % Behe-marjina
	left=2.5cm, % Ezker-marjina
	right=2.5cm, % Eskuin marjina
	headheight=14pt, % Idazpuru altuera
	footskip=1.4cm, % Beheko ertzetik plama-oineko lerroraino tartea
	headsep=1.2cm, % Goiko ertzetik idazpuruko lerroraino tartea
	%showframe, % Deskomentatu orrialdean blokea nola dagoen agertzeko
}

%----------------------------------------------------------------------------------------
%	Letra-iturri
%----------------------------------------------------------------------------------------

\usepackage[utf8]{inputenc} % Karaktere internazionalentzat beharrezkoa
\usepackage[T1]{fontenc} % Ageritzen du karaktere internazionalentzako letra-iturri kodetzea 
\usepackage[default]{lato} % Lato letra-iturri erabiltzeko

%----------------------------------------------------------------------------------------
%	Idazpuru eta plama-oinak
%----------------------------------------------------------------------------------------

\usepackage{fancyhdr} % Idazpuru eta plama-oinak egokitzeko beharrezkoak

\pagestyle{fancy} % Gaitu egokitako idazpuru eta plama-oinak

\lhead{\small\assignmentClass\ - \assignmentTitle} % Ezker-idazpuru; Ageri instruktorea kakotzen artean ezarrita egoterakoan
\chead{} % Erdiko idazpuru
\rhead{\small\ifdef{\assignmentAuthorName}{\assignmentAuthorName}{\ifdef{\assignmentDueDate}{Due\ \assignmentDueDate}{}}} % Eskumako idazpuru; Ageri egile-izena ezarrita bazegoen, bestela data ezarrita egoterakoan

%\lfoot{} % Ezker plama-oin
\cfoot{} % Erdiko plama-oin
\rfoot{\small Página\ \thepage\ -\ \pageref{LastPage}} % Eskumako plama-oin

\renewcommand\headrulewidth{0.5pt} % Idazpuruaren lerroko loditasuna

%----------------------------------------------------------------------------------------
%	Ataleko estiloa alderatu
%----------------------------------------------------------------------------------------

\usepackage{titlesec} % Atalak alderatzeko beharrezkoa

%------------------------------------------------
% Atala

\titleformat
{\section} % Atala alderatze
[block] % Mota hauetakoak izan daitezke: eseki, blokeatu, erakutsi, errunin, ezkutu, rightmargin, tanta, tanta, markoa
        % hang, block, display, runin, leftmargin, rightmargin, drop, wrap, frame
{\Large\bfseries} % Atal osoko formatua
{\assignmentQuestionName~\thesection} % Atal labeleko formatua
{6pt} % Izenburutik labeleraino tartea
{} % Label baino lehenagoko kodea

\titlespacing{\section}{0pt}{0.5\baselineskip}{0.5\baselineskip} % Atal izenburuen inguruko tarteak, ordena hau da: ezkerrera, aurretik eta ondoren

%------------------------------------------------
% Azpiatala

\titleformat
{\subsection} % Azpiatala alderatze
[block] % Mota hauetakoak izan daitezke: eseki, blokeatu, erakutsi, errunin, ezkutu, rightmargin, tanta, tanta, markoa
        % hang, block, display, runin, leftmargin, rightmargin, drop, wrap, frame
{\itshape} % Atal osoko formatua
{(\alph{subsection})} % Atal labeleko formatua
{4pt} % Izenburutik labeleraino tartea
{} % Label baino lehenagoko kodea

\titlespacing{\subsection}{0pt}{0.5\baselineskip}{0.5\baselineskip} % Atal izenburuen inguruko tarteak, ordena hau da: ezkerrera, aurretik eta ondoren

\renewcommand\thesubsection{(\alph{subsection})}

%----------------------------------------------------------------------------------------
%	Ingurumen/komando pertsonalizatuak 
%----------------------------------------------------------------------------------------

%Forzar elemento en el indice {tipo de elemento}{nombre}
\newcommand{\palindice}[2]{\addcontentsline{toc}{#1}{#2}}

%------------------------------------------------

% Laneko galderak jartzeko ingurumena
\newenvironment{question}{
	\vspace{0.5\baselineskip} % Zuriunea galderaren aurretik
	\section{} % Atalaren izenburu hutsa (adib. Galdera 2 baino ez)
	%\palindice{section}{Primera Pregrunta}
	\lfoot{\small\itshape\assignmentQuestionName~\thesection~continua en la siguiente página\ldots} % Set the left footer to state the question continues on the next page, this is reset to nothing if it doesn't (below)
}{
	\lfoot{} % Reset the left footer to nothing if the current question does not continue on the next page
}

%------------------------------------------------

% Environment for subquestions, takes 1 argument - the name of the section
\newenvironment{subquestion}[1]{
	\subsection{#1}
}{
}

\newenvironment{subsubquestion}[1]{
    \hspace{}
	\subsubsection{#1}
}{
}

%------------------------------------------------

% Environment for subquestions, takes 1 argument - the name of the section (Markoztatutako Subquestion
\newenvironment{mksubquestion}[1]{
	\subsection{\markoztatu{#1}}
}{
}

%------------------------------------------------

% Command to print a question sentence
\newcommand{\questiontext}[1]{
	\begin{gtmarkoztatu}
	\text{}{#1}
	\end{gtmarkoztatu}
	\vspace{0.5\baselineskip} % Whitespace afterwards
}

% Version con mayus
\newcommand{\questionText}[1]{
	\begin{gtmarkoztatu}
	\text{}{#1}
	\end{gtmarkoztatu}
	\vspace{0.5\baselineskip} % Whitespace afterwards
}
%------------------------------------------------

% Command to print a box that breaks across pages with the question answer
\newcommand{\answer}[1]{
	\begin{tcolorbox}[breakable, enhanced]
		#1
	\end{tcolorbox}
}

%------------------------------------------------

% Command to print a box that breaks across pages with the space for a student to answer
\newcommand{\answerbox}[1]{
	\begin{tcolorbox}[breakable, enhanced]
		\vphantom{L}\vspace{\numexpr #1-1\relax\baselineskip} % \vphantom{L} to provide a typesetting strut with a height for the line, \numexpr to subtract user input by 1 to make it 0-based as this command is
	\end{tcolorbox}
}

%------------------------------------------------

% Command to print an assignment section title to split an assignment into major parts
\newcommand{\assignmentSection}[1]{
	{
		\centering % Centre the section title
		\vspace{2\baselineskip} % Whitespace before the entire section title
		
		\rule{0.8\textwidth}{0.5pt} % Horizontal rule
		
		\vspace{0.75\baselineskip} % Whitespace before the section title
		{\LARGE \MakeUppercase{#1}} % Section title, forced to be uppercase
		
		\rule{0.8\textwidth}{0.5pt} % Horizontal rule
		
		\vspace{\baselineskip} % Whitespace after the entire section title
	}
}

%------------------------------------------------

% Zeozer markoztatzeko komandoa (edukira doitu) (Gogoratu // jarri behar dela)
\newcommand{\markoztatu}[1]{\fboxrule=0.5pt \fboxsep=6pt\fbox{#1}}
\newcommand{\enmarcar}[1]{\fboxrule=0.5pt \fboxsep=6pt\fbox{#1}}

% Zeozer markoztatzeko ingurumena (orrira doitu)
\newsavebox{\kutxan} % gtmarkoztatu (guztiz markoztatu) -ko ingurumenaren definizioa
\newenvironment{gtmarkoztatu}
    {\begin{center} \begin{lrbox}{\kutxan}\begin{minipage}{.98\textwidth}}
    {\end{minipage}\end{lrbox}\fboxrule=1pt \fboxsep=6pt\fbox{\usebox{\kutxan}} \end{center} }

%------------------------------------------------

%Figura (Con imagen (una) ) {escala}{path}{caption}
\newcommand{\irudi}[3]{
    \vspace{0.3cm}
    \begin{figure}[!h]
    	\centering
    	\includegraphics[#1]{img/#2}
    	\ifblank{#3}{}{\caption{#3}}
    	\label{fig:01} % Deskomentatu agertzeko irudi-zenbakia
    \end{figure}
    \hfill
    \vspace{0.5cm}
}

\newcommand{\irudisuelto}[3]{
\tcbset{colframe=blue!50!black,colback=white,colupper=red!50!black,fonttitle=\bfseries,nobeforeafter,center title}
                \tcbox[left=0mm,right=0mm,top=0mm,bottom=0mm,boxsep=1mm,arc=0mm,boxrule=0.5pt,title=#3]
                {\includegraphics[#1]{img/#2}}
}

%------------------------------------------------

\newcommand{\PATO}{
    \noindent\makebox[\linewidth]{\rule{\paperwidth}{10pt}}
}

% Zerredatuak

% \newcommand{zerrMarkoztatu}[1]{
%     \begin{tcblisting}{colback=black!5,colframe=black!50!black,title=Declaración de la zona y del archivo de zona, listing only,hbox,enhanced,before=\begin{center},after=\end{center},listing options={numbers=left,numbersep=5pt,numberstyle=\tiny,numberfirstline = false,breaklines=true,language=C}}
%     {#1}
%     \end{tcblisting}   
% }

% \lstset{ frame=Ltb,
% framerule=0pt,
% aboveskip=0.5cm,
% framextopmargin=3pt,
% framexbottommargin=3pt,
% framexleftmargin=0.2cm,
% framexrightmargin=0.2cm,
% framesep=0pt,
% rulesep=.4pt,
% backgroundcolor=\color{gray97},
% rulesepcolor=\color{black},
% %
% stringstyle=\ttfamily,
% showstringspaces = false,
% basicstyle=\small\ttfamily,
% commentstyle=\color{gray45},
% keywordstyle=\bfseries,
% %
% numbers=left,
% numbersep=15pt,
% numberstyle=\tiny,
% numberfirstline = false,
% breaklines=true,
% }

% \lstnewenvironment{listing}[1][]
% {\lstset{#1}\pagebreak[0]}{\pagebreak[0]}

% \lstdefinestyle{consola}
% {basicstyle=\scriptsize\bf\ttfamily,
% backgroundcolor=\color{gray75},
% }

% \lstdefinestyle{C}
% {language=C,
% }
% \newenvironment{irudi}[2]{
%     \vspace{#1}
%     #1
%     \vspace{#2}
% }

% %Figura (Con imágenes (más de una) )
% \vspace{0.3cm}
% \begin{figure}[!h]
%     \centering
%     \subfloat[Mínimos]{
%         \includegraphics[width=0.5\textwidth]{cinco/cincoWolUno.PNG}
%         }
%     \subfloat[Máximos]{
%         \includegraphics[width=0.5\textwidth]{cinco/cincoWolDos.PNG}
%     }
%     \newline
%     \subfloat[Puntos de Inflexión]{
%         \includegraphics[width=0.5\textwidth]{cinco/cincoWolTres.PNG}
%     }
%  \caption{Máximos, mínimos y puntos de inflexión de la función f(x) en WolframAlpha}
%  \label{fig:01}
% \end{figure}
% \hfill
% \vspace{0.5cm}
% %Otro ejemplo multi-imagen
% \vspace{0.3cm}
% \begin{figure}[!h]
%     \centering
%     \subfloat[Área aritmética]{
%         \includegraphics[width=0.5\textwidth]{nueve/9Waritmetica.PNG}
%         }
%     \subfloat[Área geométrica]{
%         \includegraphics[width=0.5\textwidth]{nueve/9Wgeometrica.PNG}
%     }
%  \caption{Área aritmética y geométrica en WolframAlpha}
%  \label{fig:01}
% \end{figure}
% \hfill
% \vspace{0.5cm}

%----------------------------------------------------------------------------------------
%	Izenburu orrialdea
%----------------------------------------------------------------------------------------

% \author{\textbf{\assignmentAuthorName}} % Set the default title page author field
% \date{} % Don't use the default title page date field

% \title{
% 	\thispagestyle{empty} % Suppress headers and footers
% 	\vspace{0.2\textheight} % Whitespace before the title
% 	\textbf{\assignmentClass:\ \assignmentTitle}\\[-4pt]
% 	\ifdef{\assignmentDueDate}{{\small Due\ on\ \assignmentDueDate}\\}{} % If a due date is supplied, output it
% 	\ifdef{\assignmentClassInstructor}{{\large \textit{\assignmentClassInstructor}}}{} % If an instructor is supplied, output it
% 	\vspace{0.32\textheight} % Whitespace before the author name
% }
